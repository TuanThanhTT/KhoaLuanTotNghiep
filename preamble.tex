%========================== ĐỊNH DẠNG CHƯƠNG CHUẨN ĐHĐT ==========================
% Tiêu đề các phần đặc biệt: MỞ ĐẦU, KẾT LUẬN, TÀI LIỆU THAM KHẢO
\newcommand{\specialchapter}[1]{%
  \cleardoublepage
  \thispagestyle{empty}
  \vspace*{2cm}
  {\centering\normalfont\fontsize{14}{18}\bfseries #1\par}
  \vspace{2cm}
  \addcontentsline{toc}{chapter}{#1}%
  \cleardoublepage
  \thispagestyle{fancy}
}

%========================== TRONG DOCUMENT ==========================
\begin{document}
\frontmatter
\pagenumbering{gobble}
\include{trang-bia}
\include{trang-bia-phu}
\cleardoublepage

\pagenumbering{roman}
\pagestyle{fancy}
\include{loi-cam-doan}
\include{loi-cam-on}
\tableofcontents
\include{danh-muc-viet-tat}
\include{danh-muc-bang-hinh}

%=================== PHẦN CHÍNH ===================
\cleardoublepage
\mainmatter           % ← chuyển sang số Ả-rập
\pagestyle{fancy}     % ← QUAN TRỌNG: bật lại fancyhdr

% MỞ ĐẦU
\specialchapter{MỞ ĐẦU}
\chapter*{PHẦN MỞ ĐẦU}
\addcontentsline{toc}{chapter}{PHẦN MỞ ĐẦU}

\hspace*{\parindent}Trong bối cảnh trí tuệ nhân tạo và học sâu đang phát triển mạnh mẽ, bài toán phân loại ảnh thực vật nói chung và nhận diện loài hoa nói riêng đã và đang thu hút sự quan tâm lớn từ cộng đồng nghiên cứu nhờ tính ứng dụng cao trong bảo tồn đa dạng sinh học, du lịch sinh thái, giáo dục và thương mại điện tử. Tại Việt Nam, làng hoa Sa Đéc (tỉnh Đồng Tháp) là trung tâm sản xuất hoa kiểng lớn nhất khu vực Đồng bằng sông Cửu Long với diện tích hơn 950 ha, hơn 4.000 hộ trồng hoa, cung ứng trên 12 triệu sản phẩm mỗi năm và tạo ra giá trị kinh tế gần 3.349 tỷ đồng (năm 2022). Tuy nhiên, việc nhận diện và quản lý các loài hoa hiện nay vẫn chủ yếu dựa vào kinh nghiệm thủ công, chưa có hệ thống tự động hóa đáng kể.

Các bộ dữ liệu hoa phổ biến trên thế giới như Oxford-17 Flowers, Oxford-102 Flowers hay Flower-102 đều được thu thập trong điều kiện phòng thí nghiệm hoặc môi trường nước ngoài, không phản ánh đúng đặc điểm hình thái, màu sắc và điều kiện chụp thực tế của các loài hoa Việt Nam, đặc biệt là các giống hoa đặc trưng tại làng hoa Sa Đéc. Cho đến nay, tại Việt Nam vẫn chưa có bộ dữ liệu ảnh hoa công khai, chất lượng cao và được đánh giá toàn diện bởi các mô hình học sâu hiện đại.

Xuất phát từ thực trạng trên, đồ án tập trung xây dựng bộ dữ liệu ảnh hoa Việt Nam đầu tiên có tính đại diện cao, đồng thời khảo sát hiệu quả của các kiến trúc học sâu tiên tiến nhằm tìm ra giải pháp nhận diện chính xác, nhanh và khả thi cho triển khai thực tế tại địa phương.

Mục đích nghiên cứu: Xây dựng bộ dữ liệu ảnh gồm 28 loài hoa đặc trưng Việt Nam (trong đó có 11 loài tự thu thập thực địa tại làng hoa Sa Đéc), đồng thời đánh giá toàn diện năm kiến trúc học sâu hiện đại nhất năm 2025 nhằm lựa chọn mô hình tối ưu về độ chính xác, tốc độ suy luận và khả năng triển khai trên thiết bị di động và nhúng.

Nhiệm vụ nghiên cứu:
\begin{itemize}
\item Thu thập, làm sạch và xây dựng bộ dữ liệu ảnh hoa gồm 3.376 ảnh thuộc 28 loài, trong đó 11 loài được chụp thực địa tại làng hoa Sa Đéc.
\item Thực hiện các kỹ thuật tiền xử lý và tăng cường dữ liệu phù hợp với đặc điểm ảnh thực tế.
\item Huấn luyện và tinh chỉnh năm mô hình học sâu: YOLOv8n-cls, Vision Transformer (ViT-Base), ResNet50, EfficientNet-B2 Ensemble và MobileNetV2.
\item Đánh giá hiệu suất mô hình theo các chỉ số Accuracy, Macro F1, tốc độ suy luận, kích thước mô hình và khả năng giải thích bằng Grad-CAM.
\item So sánh, phân tích và đề xuất mô hình tối ưu nhất cho ứng dụng thực tế.
\end{itemize}

Đối tượng và phạm vi nghiên cứu:
\begin{itemize}
\item Đối tượng nghiên cứu: 28 loài hoa phổ biến tại Việt Nam, bao gồm 11 loài tự thu thập thực địa tại làng hoa Sa Đéc (Hoa giấy, Cúc lá nhám, Hoa bướm hồng, Hoa quỳnh anh, Hoa quỳnh anh tím, Hoa tigôn đỏ, Hoa thiên điểu, Hoa mua tím, Hoa sao nhái vàng, Dây hồng anh, Cẩm tú cầu) và 17 loài còn lại từ nguồn công cộng đã được kiểm chứng chất lượng.
\item Phạm vi nghiên cứu: Ảnh tĩnh chụp trong điều kiện ánh sáng tự nhiên, kích thước chuẩn hóa 224×224 pixel, không bao gồm ảnh chụp đêm, ảnh hồng ngoại hoặc video.
\end{itemize}

Phương pháp nghiên cứu:
\begin{itemize}
\item Phương pháp lý thuyết: Tổng quan các công trình nghiên cứu liên quan và cơ sở lý thuyết của các kiến trúc CNN, Transformer và YOLO-cls.
\item Phương pháp thực nghiệm: Sử dụng Python, PyTorch 2.6 và Ultralytics YOLOv8; huấn luyện trên cụm GPU Tesla T4 (Kaggle); áp dụng transfer learning, các kỹ thuật augmentation nâng cao (MixUp, CutMix, RandomErasing, GaussianBlur), label smoothing, focal loss, ensemble và Grad-CAM.
\item Phương pháp phân tích: Đánh giá đồng thời năm mô hình trên cùng bộ dữ liệu, cùng seed ngẫu nhiên và cùng điều kiện phần cứng nhằm đảm bảo tính công bằng và khả năng tái lập.
\end{itemize}

Những đóng góp chính của đồ án:
\begin{itemize}
\item Xây dựng và công bố bộ dữ liệu ảnh hoa Việt Nam đầu tiên gồm 28 loài với 3.376 ảnh chất lượng cao, trong đó 11 loài được thu thập thực địa tại làng hoa Sa Đéc, nguồn tài nguyên mở phục vụ cộng đồng nghiên cứu.
\item Đánh giá toàn diện năm kiến trúc học sâu hiện đại năm 2025, đạt kết quả cao: mô hình Vision Transformer (ViT-Base) được đề xuất là giải pháp tối ưu nhất nhờ độ chính xác 98,76 \%, khả năng giải thích vượt trội qua cơ chế attention và Grad-CAM.
\end{itemize}

Cấu trúc của đồ án: Ngoài phần Mở đầu, Kết luận và kiến nghị, Tài liệu tham khảo, Phụ lục, đồ án gồm bốn chương chính:
\begin{itemize}
\item Chương 1. Cơ sở lý thuyết và tổng quan nghiên cứu
\item Chương 2. Xây dựng tập dữ liệu và tiền xử lý ảnh
\item Chương 3. Thực nghiệm và đánh giá hiệu suất các mô hình
\item Chương 4. Triễn khai hệ thống và xây dựng ứng dụng
\end{itemize}

% Các chương nội dung (có chữ “Chương 1” tự động)
\chapter{Cơ sở lý thuyết và tổng quan nghiên cứu}
% ==============================================================
%  chapters/chapter1.tex
%  Chương 1: CƠ SỞ LÝ THUYẾT VÀ TỔNG QUAN NGHIÊN CỨU
%  ĐÃ ĐƯỢC SỬA HOÀN CHỈNH – Ảnh hiển thị 100%
% ==============================================================

\chapter{CƠ SỞ LÝ THUYẾT VÀ TỔNG QUAN NGHIÊN CỨU}
\label{chap:cosolythuyet}

Chương này trình bày cơ sở lý thuyết của các kiến trúc học sâu hiện đại được sử dụng trong bài toán phân loại ảnh, tập trung vào hai hướng tiếp cận chính: mạng nơ-ron tích chập (Convolutional Neural Networks -- CNN) và kiến trúc Transformer. Đồng thời, chương tổng quan các nghiên cứu tiêu biểu về nhận diện loài hoa, từ đó xác định khoảng trống nghiên cứu và định hướng giải quyết của đồ án.

\section{Bài toán phân loại ảnh và nhận diện loài hoa}

\subsection{Khái niệm}

\hspace*{\parindent}Phân loại ảnh là một trong những nhiệm vụ cốt lõi của thị giác máy tính, nhằm gán nhãn lớp cho một ảnh đầu vào. Trong bài toán nhận diện loài hoa, mục tiêu là xây dựng hàm ánh xạ từ không gian ảnh $I \in \mathbb{R}^{H \times W \times 3}$ sang tập nhãn loài $\{1, 2, \dots, C\}$ với $C$ là số loài hoa:

\begin{equation}
    f: \mathbb{R}^{H \times W \times 3} \to \{1, 2, \dots, C\}
    \label{eq:classification}
\end{equation}

\subsection{Ý nghĩa khoa học và thực tiễn}

\hspace*{\parindent}Việc tự động nhận diện loài hoa có giá trị lớn trong phân loại thực vật học, giám sát đa dạng sinh học, bảo tồn nguồn gen và hỗ trợ giáo dục cộng đồng. Tại Việt Nam, đặc biệt là làng hoa Sa Đéc ứng dụng nhận diện hoa tự động sẽ góp phần quảng bá du lịch, hỗ trợ thương mại điện tử và xây dựng cơ sở dữ liệu thực vật số.

Về mặt khoa học, đồ án đóng góp một bộ dữ liệu mới thu thập thực địa tại làng hoa Sa Đéc, giúp lấp đầy khoảng trống về dữ liệu hoa Việt Nam trong cộng đồng nghiên cứu học thuật quốc tế.

\section{Tổng quan các nghiên cứu liên quan}

\hspace*{\parindent}Các nghiên cứu nhận diện loài hoa chủ yếu được thực hiện trên hai bộ dữ liệu chuẩn Oxford-17 và Oxford-102 Flowers (102 loài, 8.189 ảnh) \cite{nilsback2008automated}. Một số công trình nghiên cứu tiêu biểu:

\begin{itemize}
    \item Mete và Ensari \cite{mete2019} đề xuất phương pháp lai giữa CNN trích xuất đặc trưng kết hợp các bộ phân loại truyền thống (SVM, KNN, Random Forest), trong đó CNN+SVM cho kết quả tốt nhất.
    \item Alipour và cộng sự \cite{alipour2021} sử dụng transfer learning với DenseNet121, đạt 98,6\% độ chính xác trên Oxford-102 sau 50 epoch.
    \item Gupta và cộng sự \cite{gupta2023} là một trong những nghiên cứu đầu tiên áp dụng Vision Transformer (ViT) cho bài toán phân loại hoa, chứng minh ViT vượt trội hơn CNN khi background phức tạp và dữ liệu đa dạng.
\end{itemize}

Do đó, vẫn còn thiếu các nghiên cứu đánh giá đồng thời các kiến trúc học sâu tiên tiến nhất trên dữ liệu thực tế có background phức tạp, điều kiện ánh sáng không kiểm soát -- đặc trưng của ảnh chụp ngoài thực địa tại các làng hoa Việt Nam.

\section{Cơ sở lý thuyết các kiến trúc học sâu}
\label{sec:cosolythuyet}

\subsection{Mạng nơ-ron tích chập (CNN) và các biến thể}

\hspace*{\parindent}Mạng CNN là kiến trúc chủ đạo trong phân loại ảnh từ năm 2012 đến nay nhờ khả năng trích xuất đặc trưng không gian phân cấp thông qua các tầng tích chập và pooling.

\begin{itemize}
    \item \textbf{ResNet50} \cite{he2016deep}: Mạng nơ-ron tích chập sâu sử dụng kết nối tắt (residual connection) để khắc phục hiện tượng vanishing gradient. Nhờ đó, các biến thể như ResNet50, ResNet101, ResNet152 có thể được huấn luyện ổn định và đạt hiệu suất vượt trội trên ImageNet.
    \item \textbf{MobileNetV2} \cite{sandler2018mobilenetv2}: Sử dụng depthwise separable convolution và inverted residuals, giảm đáng kể số tham số, phù hợp thiết bị di động.
    \item \textbf{EfficientNet-B2} \cite{tan2019efficientnet}: Một biến thể của dòng EfficientNet, được phát triển dựa trên nguyên tắc compound scaling đồng bộ ba yếu tố: độ sâu, độ rộng và độ phân giải đầu vào. Với chỉ 9 triệu tham số, EfficientNet-B2 vượt trội ResNet50 (25,6 triệu tham số) về độ chính xác trên ImageNet đồng thời giảm đáng kể chi phí tính toán, là lựa chọn lý tưởng trong chiến lược ensemble của đồ án.
 \end{itemize}

\begin{figure}[H]
    \centering
    % Cách an toàn nhất: dùng dấu nháy kép + dấu gạch chéo xuôi
    \includegraphics[width=0.98\linewidth]{figures/chapter1/mo_hinh_cnn.jpg}
    \caption{Kiến trúc tổng quát của mạng nơ-ron tích chập (CNN)}
    \label{fig:cnn_arch}
\end{figure}

\subsection{Kiến trúc Transformer và Vision Transformer (ViT)}

\hspace*{\parindent}Transformer được Vaswani và cộng sự đề xuất năm 2017 cho xử lý ngôn ngữ tự nhiên \cite{vaswani2017attention}, đánh dấu bước ngoặt khi thay thế hoàn toàn các tầng recurrent bằng cơ chế tự chú ý (self-attention). Năm 2020, Dosovitskiy và cộng sự mở rộng kiến trúc này sang lĩnh vực thị giác với Vision Transformer (ViT) \cite{dosovitskiy2020image}, chứng minh Transformer có thể cạnh tranh và vượt trội CNN khi được huấn luyện trên dữ liệu quy mô lớn.

ViT hoạt động theo nguyên lý sau:

\textbf{1. Patch Embedding:} Ảnh đầu vào $X \in \mathbb{R}^{H \times W \times C}$ được chia thành $N = \frac{HW}{P^2}$ patch có kích thước $P \times P$ (thường $P=16$). Mỗi patch $x_p \in \mathbb{R}^{P^2 \cdot C}$ được làm phẳng và chiếu tuyến tính thành vector nhúng $z_p \in \mathbb{R}^D$:
\begin{equation}
    z_p = x_p E, \quad E \in \mathbb{R}^{(P^2 \cdot C) \times D}
    \label{eq:patch_embedding}
\end{equation}

\textbf{2. Class Token và Positional Embedding:} Một token học được $z_{cls} \in \mathbb{R}^D$ được thêm vào đầu chuỗi, kèm theo positional embedding $E_{pos} \in \mathbb{R}^{(N+1) \times D}$ để bảo toàn thông tin vị trí:
\begin{equation}
    z_0 = [z_{cls}; z_{p_1}; z_{p_2}; \dots; z_{p_N}] + E_{pos}
    \label{eq:transformer_input}
\end{equation}

\textbf{3. Transformer Encoder:} Chuỗi $z_0$ được đưa qua $L$ tầng Transformer encoder. Mỗi tầng bao gồm Multi-Head Self-Attention (MSA) và MLP với kết nối tắt:
\begin{align}
    Z'_\ell &= \text{MSA}(\text{LayerNorm}(Z_{\ell-1})) + Z_{\ell-1} \label{eq:transformer_msa} \\
    Z_\ell  &= \text{MLP}(\text{LayerNorm}(Z'_\ell)) + Z'_\ell \label{eq:transformer_mlp}
\end{align}

\textbf{4. Phân loại:} Vector $z_{cls}$ từ tầng cuối $Z_L$ được đưa qua MLP Head để dự đoán $C$ lớp:
\begin{equation}
    \hat{y} = \text{Softmax}(W z_L^{cls})
    \label{eq:vit_classification}
\end{equation}

ViT chứng minh Transformer có khả năng học quan hệ toàn cục tốt hơn CNN khi được huấn luyện trên dữ liệu lớn (JFT-300M), đạt top-1 accuracy 88,55\% trên ImageNet với ViT-Huge/14 -- vượt qua mọi CNN cùng quy mô.

\begin{figure}[H]
    \centering
    \includegraphics[width=0.98\linewidth]{figures/chapter1/mo_hinh_vit.jpg}
    \caption{Kiến trúc tổng quát của Vision Transformer (ViT)}
    \label{fig:vit_arch}
\end{figure}

ViT chứng minh khả năng học quan hệ toàn cục (global context) tốt hơn CNN, đặc biệt hiệu quả khi dữ liệu huấn luyện lớn và đa dạng.

\subsection{YOLOv8n-cls -- Kiến trúc phân loại tốc độ cao}

\hspace*{\parindent}YOLOv8n-cls là phiên bản chuyên biệt cho bài toán phân loại ảnh của dòng mô hình YOLOv8, được Ultralytics chính thức phát hành vào tháng 01/2023 \cite{ultralytics2023}. Khác với các biến thể object detection, YOLOv8n-cls được thiết kế tối giản bằng cách loại bỏ hoàn toàn các nhánh dự đoán bounding box và anchor, chỉ giữ lại đường ống trích xuất đặc trưng và đầu phân loại toàn cục.

Kiến trúc của YOLOv8n-cls bao gồm các thành phần chính sau:
\begin{itemize}
    \item \textbf{Backbone CSPDarknet hiện đại}: Sử dụng các khối \texttt{C2f} (CSP Bottleneck với 2 phép tích chập) kết hợp hàm kích hoạt SiLU, giúp tăng cường luồng gradient và giảm đáng kể chi phí tính toán so với các backbone truyền thống.
    \item \textbf{Neck nhẹ dựa trên PAN}: Tổng hợp đặc trưng đa tỷ lệ một cách hiệu quả để tạo biểu diễn toàn cục mạnh mẽ cho nhiệm vụ phân loại.
    \item \textbf{Classification head}: Một tầng fully-connected kết thúc bằng hàm Softmax (hoặc sigmoid trong chế độ multi-label) để đưa ra dự đoán xác suất lớp.
\end{itemize}

Với chỉ \textbf{3,1 triệu tham số} và \textbf{8,7 GFLOPs}, YOLOv8n-cls đạt độ chính xác top-1 là 69,0\% trên ImageNet-1k (pretrained), đồng thời sở hữu tốc độ suy luận cực nhanh: khoảng 0,31 ms/ảnh trên GPU A100 (TensorRT) và dưới 13 ms trên CPU thông thường. Những ưu điểm vượt trội về tốc độ và hiệu suất tài nguyên khiến YOLOv8n-cls trở thành lựa chọn lý tưởng cho các ứng dụng thực tế yêu cầu triển khai trên thiết bị nhúng hoặc hệ thống thời gian thực -- một trong những tiêu chí quan trọng của đồ án này.

\subsection{Các kỹ thuật hỗ trợ huấn luyện quan trọng}
\label{sec:training_techniques}

\hspace*{\parindent}Để đạt được hiệu suất cao trên tập dữ liệu thực địa có số lượng mẫu hạn chế và phân bố lớp không cân bằng, đồ án áp dụng đồng thời nhiều kỹ thuật huấn luyện tiên tiến đã được cộng đồng học sâu công nhận:

\begin{itemize}
    \item \textbf{Học chuyển giao (Transfer Learning)} \cite{pan2010survey, yosinski2014transferable}: 
    Tất cả các mô hình đều được khởi tạo trọng số từ quá trình huấn luyện trước trên ImageNet-1k hoặc ImageNet-21k. Phương pháp này giúp mạng hội tụ nhanh hơn và đạt độ chính xác cao hơn đáng kể so với huấn luyện từ đầu (training from scratch), đặc biệt hiệu quả khi dữ liệu huấn luyện chỉ vài nghìn ảnh mỗi lớp.

    \item \textbf{Tăng cường dữ liệu mạnh (Strong Data Augmentation)}:
    Kết hợp nhiều phép biến đổi ngẫu nhiên đồng thời bao gồm RandomResizedCrop, RandomHorizontalFlip, ColorJitter, GaussianBlur, cùng các kỹ thuật trộn mẫu cấp cao MixUp \cite{zhang2018mixup} và CutMix \cite{yun2019cutmix}. Những phép tăng cường này làm tăng tính đa dạng của dữ liệu, giảm hiện tượng quá khớp và cải thiện khả năng khái quát hóa trên ảnh thực địa có điều kiện ánh sáng và góc chụp thay đổi lớn.

    \item \textbf{Xử lý mất cân bằng lớp (Class Imbalance Handling)}:
    Sử dụng Weighted Random Sampler trong quá trình lấy batch hoặc áp dụng Focal Loss \cite{lin2017focal} để tập trung huấn luyện nhiều hơn vào các lớp hiếm (ví dụ: một số loài hoa chỉ có dưới 200 ảnh), từ đó nâng cao độ nhạy (recall) trên toàn bộ các lớp.

    \item \textbf{Tối ưu hóa và điều độ học suất hiện đại}:
    Bộ tối ưu AdamW \cite{loshchilov2019decoupled} kết hợp lịch trình Cosine Annealing with Warm Restarts \cite{loshchilov2017sgdr} và kỹ thuật Label Smoothing \cite{szegedy2016rethinking} được sử dụng nhằm ổn định quá trình huấn luyện, tránh cực tiểu cục bộ và cải thiện độ chính xác cuối cùng từ 0,5--1,2\%.

    \item \textbf{Trực quan hóa và giải thích mô hình}:
    Grad-CAM \cite{selvaraju2017grad} được áp dụng sau huấn luyện để sinh heatmap chỉ ra vùng ảnh mà mô hình tập trung khi đưa ra dự đoán. Kỹ thuật này không chỉ giúp đánh giá tính hợp lý của mô hình mà còn là cơ sở để phát hiện các trường hợp học sai (ví dụ: mô hình chỉ nhìn vào lá thay vì hoa).
\end{itemize}


\chapter{Xây dựng bộ dữ liệu và tiền xử lý ảnh}
\chapter{XÂY DỰNG TẬP DỮ LIỆU VÀ TIỀN XỬ LÝ ẢNH}
\label{chap:dataset}

Chương này trình bày chi tiết nguồn gốc, quy mô và đặc điểm của tập dữ liệu được sử dụng trong đồ án. Đồng thời mô tả toàn bộ quy trình tiền xử lý dữ liệu gồm làm sạch, chuẩn hóa kích thước ảnh, chia tập huấn luyện - kiểm định - kiểm tra và các kỹ thuật tăng cường dữ liệu nhằm nâng cao khả năng khái quát hóa của mô hình.

\section{Nguồn gốc và quy mô tập dữ liệu}

\hspace*{\parindent}Tập dữ liệu mang tên \textbf{VietnamFlowers-28} được xây dựng nhằm phục vụ nhiệm vụ nhận diện 28 loài hoa phổ biến tại Việt Nam, trong đó có 11 loài do chính tác giả tự thu thập thực địa tại làng hoa Sa Đéc tỉnh Đồng Tháp. Các loài còn lại được lấy từ tập dữ liệu công khai nổi tiếng Oxford 102 Flowers, sau đó được chọn lọc kỹ lưỡng để phù hợp với bài toán đặt ra.
\begin{table}[htbp]
\centering
\caption{Thống kê tổng quan tập dữ liệu VietnamFlowers-28}
\label{tab:dataset_summary}
\begin{tabular}{lr}
\toprule
\textbf{Tiêu chí}                  & \textbf{Giá trị}                  \\
\midrule
Tổng số ảnh                        & 3.376                            \\
Số loài hoa                        & 28                               \\
Số ảnh trung bình mỗi lớp          & ~120                             \\
Số ảnh ít nhất mỗi lớp             & 50 (Hoa cẩm chướng)              \\
Số ảnh nhiều nhất mỗi lớp          & 234 (Dạ yến thảo)                \\
Định dạng ảnh                      & JPEG                             \\
Độ phân giải gốc                   & 512 × 512 đến 4032 × 3024 pixel  \\
Điều kiện chụp                     & Ánh sáng tự nhiên, ngược sáng \\
\bottomrule
\end{tabular}
\end{table}

Danh sách chi tiết 28 loài và số lượng ảnh từng loài được trình bày ở Bảng \ref{tab:28species}.

\subsection{Các thách thức của tập dữ liệu thực địa}
\label{subsec:challenges}

\hspace*{\parindent}So với các bộ dữ liệu chuẩn được thu thập trong môi trường kiểm soát như Oxford 102 Flower Dataset \cite{flower102}, tập dữ liệu VietnamFlowers-28 thể hiện đầy đủ những khó khăn đặc trưng của dữ liệu thực địa tại Việt Nam, cụ thể như sau:

\begin{enumerate}
    \item \textbf{Biến thiên nội lớp lớn (large intra-class variation)}:  
    Cùng một loài hoa nhưng xuất hiện ở các giai đoạn sinh trưởng khác nhau (nụ, hé nở, nở bung), nhiều góc chụp (từ trên xuống, ngang, cận cảnh, toàn cây), và dưới các điều kiện thời tiết.
    
    \item \textbf{Tương đồng liên lớp cao (high inter-class similarity)}:  
    Một số loài có ngoại hình rất gần nhau, dễ gây nhầm lẫn ngay cả với con người, ví dụ: các giống cúc vạn thọ, quỳnh anh, quỳnh anh tím, hoặc các biến thể của hoa hồng môn.
    
    \item \textbf{Nền cảnh phức tạp và chứa nhiều nhiễu}:  
    Ảnh thực tế thường có nền là đất trồng, cỏ dại, lưới che nắng, chậu nhựa, bao bì, dây buộc, và đôi khi có sự xuất hiện của người hoặc vật thể khác, làm tăng độ khó trong việc trích xuất đặc trưng hoa.
    
    \item \textbf{Điều kiện chiếu sáng không đồng đều}:  
    Bao gồm ánh sáng tự nhiên mạnh (ngược sáng), bóng đổ của cây và lưới che, ánh sáng đèn nhân tạo vào ban đêm, cũng như hiện tượng phản xạ trên lá ướt sau mưa.
    
    \item \textbf{Mất cân bằng nhẹ giữa các lớp}:  
    Một số loài ít phổ biến chỉ có số lượng mẫu dưới 70 ảnh, trong khi loài phổ biến nhất đạt 234 ảnh, dẫn đến nguy cơ mô hình bị thiên lệch về các lớp chiếm ưu thế.
\end{enumerate}

Những thách thức nêu trên đòi hỏi phải áp dụng quy trình tiền xử lý và tăng cường dữ liệu mạnh mẽ, đồng thời lựa chọn kiến trúc mô hình có khả năng học đặc trưng bất biến và phân biệt tốt giữa các lớp tương đồng — đây cũng chính là động lực để thử nghiệm nhiều kiến trúc hiện đại trong các chương tiếp theo.

\begin{table}[htbp]
\centering
\caption{Danh sách 28 loài hoa trong tập dữ liệu}
\label{tab:28species}
\small
\begin{tabular}{l r | l r}
\toprule
\textbf{Tên loài}                  & \textbf{Số ảnh} & \textbf{Tên loài}                  & \textbf{Số ảnh} \\
\midrule
Hoa bướm hồng*                     & 147   & Hoa mõm sói                       & 78    \\
Hoa cẩm chướng                     & 50    & Hoa mua tím*                      & 149   \\
Hoa cẩm tú cầu*                    & 141   & Hoa ngọc lan                      & 57    \\
Hoa cúc đồng tiền Châu Phi         & 57    & Hoa ngũ sắc Nam Phi               & 69    \\
Hoa cúc lá nhám*                   & 145   & Hoa quỳnh anh*                    & 155   \\
Hoa cúc vạn thọ                    & 55    & Hoa quỳnh anh tím*                & 145   \\
Hoa cúc vạn thọ Anh                & 58    & Hoa sao nhái vàng*                & 138   \\
Hoa dạ yến thảo                    & 234   & Hoa sen                           & 123   \\
Hoa dâm bụt                        & 120   & Hoa sứ                            & 138   \\
Hoa dây hồng anh*                  & 157   & Hoa súng                          & 176   \\
Hoa đỗ quyên                       & 83    & Hoa thiên điểu*                   & 138   \\
Hoa đồng tiền Nam Phi              & 112   & Hoa Tigôn đỏ*                     & 161   \\
Hoa giấy*                          & 146   & Hoa hồng                          & 172   \\
Hoa hồng môn*                      & 122   & Hoa hướng dương                   & 62    \\
\bottomrule
\multicolumn{4}{l}{\footnotesize{(*): 11 loài do tác giả tự thu thập thực địa tại làng hoa Sa Đéc}}
\end{tabular}
\end{table}

\section{Quy trình tiền xử lý dữ liệu}

\subsection{Làm sạch và tổ chức dữ liệu}

\hspace*{\parindent}Trước khi tiến hành các bước tiền xử lý tiếp theo, toàn bộ dữ liệu thô được thực hiện quy trình làm sạch nghiêm ngặt nhằm đảm bảo chất lượng và tính nhất quán của tập dữ liệu:

\begin{enumerate}
    \item \textbf{Kiểm tra và loại bỏ các mẫu không hợp lệ}:  
    Các ảnh bị lỗi định dạng, hỏng file, độ phân giải quá thấp hoặc bị mờ nghiêm trọng (không thể nhận diện được đối tượng chính) đã được loại bỏ hoàn toàn.
    
    \item \textbf{Phát hiện và loại bỏ ảnh trùng lặp}:  
    Sử dụng thuật toán \textit{perceptual hashing} (pHash) để tính toán giá trị băm cảm nhận của từng ảnh. Các cặp ảnh có khoảng cách Hamming nhỏ hơn hoặc bằng 5 được coi là trùng lặp và chỉ giữ lại một bản đại diện.
    
    \item \textbf{Chuẩn hóa hệ thống đặt tên}:  
          Toàn bộ ảnh được đổi tên theo cấu trúc  
          \texttt{\{Mã\_loài\}\_\{STT\}.jpg}  
          và được đặt trong thư mục tương ứng với tên loài.  
          Ví dụ: một ảnh thuộc loài hoa quỳnh anh tím sẽ có đường dẫn  
          \texttt{HoaQuynhAnhTim/HoaQuynhAnhTim\_0123.jpg}.  

    \item \textbf{Tổ chức cấu trúc thư mục chuẩn}:  
    Dữ liệu được sắp xếp theo cấu trúc phân tầng ba tập con độc lập:  
    \texttt{Train}, \texttt{Valid}, và \texttt{Test}, trong đó mỗi tập chứa 28 thư mục con tương ứng với 28 loài hoa. Cấu trúc này tuân thủ chuẩn phổ biến trong các framework học sâu hiện đại (PyTorch, TensorFlow, Keras).
\end{enumerate}

Quy trình làm sạch và tổ chức nêu trên không chỉ nâng cao chất lượng dữ liệu đầu vào mà còn tạo điều kiện thuận lợi cho các bước tiền xử lý, chia tập và huấn luyện mô hình ở các giai đoạn tiếp theo.

\subsection{Chuẩn hóa kích thước ảnh}
\label{subsec:resize}

\hspace*{\parindent}Để đảm bảo tính đồng nhất của dữ liệu đầu vào cho các mô hình học sâu, toàn bộ ảnh trong tập dữ liệu được chuyển đổi về kích thước cố định \textbf{224 × 224} pixel. Tuy nhiên, thay vì sử dụng phương pháp resize thông thường (có thể gây biến dạng tỷ lệ khung hình), đồ án áp dụng kỹ thuật \textbf{resize kết hợp padding} như sau:

\begin{enumerate}
    \item Ảnh gốc được thu nhỏ sao cho cạnh dài nhất đúng bằng 224 pixel, đồng thời giữ nguyên hoàn toàn tỷ lệ khung hình (aspect ratio). Quá trình thu nhỏ sử dụng bộ lọc \textbf{LANCZOS} — thuật toán nội suy chất lượng cao có trong thư viện PIL, giúp giảm thiểu hiện tượng răng cưa và mất chi tiết.
    
    \item Tạo một ảnh nền trắng có kích thước 224 × 224 pixel.
    
    \item Dán ảnh đã thu nhỏ vào chính giữa ảnh nền (center padding), nhờ đó các pixel bổ sung không làm thay đổi nội dung hoa mà chỉ đóng vai trò lấp đầy vùng trống.
    
    \item Toàn bộ quy trình được triển khai bằng thư viện \textbf{Pillow (PIL)} kết hợp kỹ thuật xử lý song song (\textbf{multiprocessing}) trên nền tảng Google Colab, giúp giảm đáng kể thời gian xử lý đối với hơn 3.300 ảnh.
\end{enumerate}

\begin{figure}[htbp]
\centering
\includegraphics[width=0.65\linewidth]{figures/chapter2/KyThuatResize.png}
\caption{Kỹ thuật resize kết hợp padding (trái: ảnh gốc; phải: ảnh chuẩn hóa 224 × 224).}
\label{fig:resize_padding}
\end{figure}

Phương pháp này không chỉ bảo toàn hình dáng tự nhiên của hoa mà còn giúp mô hình tập trung tốt hơn vào đối tượng chính, đồng thời tránh được hiện tượng biến dạng thường gặp khi sử dụng resize ép buộc về kích thước cố định. Kỹ thuật tương tự đã được áp dụng thành công trong nhiều nghiên cứu trước đây \cite{he2016deep,sandler2018mobilenetv2,tan2019efficientnet}.

\subsection{Chia tập dữ liệu}
\hspace*{\parindent}Sử dụng \textbf{stratified split} với tỷ lệ \textbf{70\% – 15\% – 15\%} để đảm bảo mỗi lớp có tỷ lệ ảnh giống nhau trong cả ba tập.

\begin{table}[htbp]
\centering
\caption{Phân bố dữ liệu sau khi chia (stratified)}
\label{tab:data_split}
\begin{tabular}{l r r r}
\toprule
Tập           & Số ảnh   & Tỷ lệ   & Ảnh trung bình/lớp \\
\midrule
Train         & 2.363    & 70\%    & ~84                \\
Validation    & 506      & 15\%    & ~18                \\
Test          & 507      & 15\%    & ~18                \\
\midrule
Tổng cộng     & 3.376    & 100\%   & 120                \\
\bottomrule
\end{tabular}
\end{table}


\subsection{Tăng cường dữ liệu}
\label{subsec:augmentation}

\hspace*{\parindent}Để nâng cao khả năng khái quát hóa của mô hình và giảm nguy cơ quá khớp (overfitting) trên tập dữ liệu thực địa có số lượng mẫu hạn chế, đồ án áp dụng chiến lược tăng cường dữ liệu mạnh mẽ trong quá trình huấn luyện. Các phép biến đổi được thực hiện ngẫu nhiên trên từng batch dữ liệu (on-the-fly augmentation) bằng thư viện \textbf{Albumentations} \cite{buslaev2020albumentations} kết hợp với \textbf{PyTorch transforms} \cite{pytorch2025}, giúp tăng tính đa dạng của dữ liệu mà không cần lưu trữ thêm ảnh vật lý.

Các kỹ thuật được phân loại thành hai nhóm chính:

\textbf{Nhóm biến đổi hình học (geometric transformations)} nhằm mô phỏng các biến đổi không gian thường gặp trong ảnh thực địa:
- \texttt{RandomResizedCrop} với tỷ lệ scale từ 0.7 đến 1.0 và thay đổi ngẫu nhiên tỷ lệ khung hình, giúp mô hình học bất biến với kích thước và góc nhìn khác nhau.
- \texttt{RandomHorizontalFlip} (xác suất 0,5) và \texttt{RandomVerticalFlip} (xác suất 0,3), mô phỏng việc lật ảnh ngang/dọc để tăng tính đối xứng.
- \texttt{RandomRotation} lên đến 90 độ và \texttt{RandomAffine} với dịch chuyển (translate 0.2), co giãn (scale 0.8--1.2), giúp mô hình xử lý tốt các ảnh chụp từ nhiều góc quay.

\textbf{Nhóm biến đổi màu sắc và nhiễu (pixel-level transformations)} nhằm tái tạo các điều kiện chiếu sáng và môi trường thực tế:
\texttt{ColorJitter} với biên độ thay đổi độ sáng, độ tương phản, độ bão hòa (±0,4) và sắc màu (±0,1), mô phỏng ánh sáng thay đổi do thời tiết hoặc thời gian trong ngày.
\texttt{GaussianBlur} và \texttt{MotionBlur} để tạo hiệu ứng mờ do chuyển động hoặc thời tiết xấu.
\texttt{RandomFog}, \texttt{RandomShadow}, \texttt{RandomErasing} (xác suất 0,5) để thêm sương mù, bóng đổ và che khuất ngẫu nhiên, tăng khả năng chống nhiễu cho mô hình.

Ngoài ra, các kỹ thuật trộn mẫu cấp cao như \texttt{CutMix} (\(\alpha = 1.0\)) \cite{yun2019cutmix} và \texttt{MixUp} (\(\alpha = 0.2\)) \cite{zhang2018mixup} được áp dụng để tạo ra các mẫu mới bằng cách kết hợp tuyến tính hoặc cắt dán giữa hai ảnh, đồng thời trộn nhãn tương ứng. Những phương pháp này không chỉ tăng số lượng dữ liệu hiệu quả mà còn giúp mô hình học được các đặc trưng phân biệt tốt hơn giữa các lớp tương đồng.

Chiến lược tăng cường dữ liệu nêu trên đã được chứng minh mang lại cải thiện đáng kể về độ chính xác và độ bền vững trên nhiều tập dữ liệu thực tế \cite{shorten2019survey}.

\chapter{Thực nghiệm và đánh giá các mô hình học sâu}
% ==========================
% chapters/chapter3.tex – HOÀN CHỈNH 100%, CÓ GRAD-CAM HIỂN THỊ ĐẸP, KHÔNG LỖI
% ==========================

% === TỰ ĐỘNG FIX LỖI ẢNH KHÔNG TỒN TẠI (KHÔNG CRASH, HIỂN KHUNG TRẮNG ĐẸP) ===
\let\oldincludegraphics\includegraphics
\renewcommand{\includegraphics}[2][]{%
  \IfFileExists{#2}{%
    \oldincludegraphics[#1]{#2}%
  }{%
    \fbox{%
      \begin{minipage}{0.8\textwidth}%
        \centering
        \vspace{1em}%
        \large \textcolor{gray}{[Ảnh sẽ hiện ở đây]} \\
        \small \textcolor{lightgray}{#2}%
        \vspace{1em}%
      \end{minipage}%
    }%
  }%
}

\chapter{THỰC NGHIỆM VÀ ĐÁNH GIÁ HIỆU SUẤT CÁC MÔ HÌNH}
\label{chap:experiments}

Chương này trình bày chi tiết môi trường thực nghiệm, quy trình huấn luyện chung, kết quả định lượng và phân tích định tính của năm mô hình học sâu trên tập dữ liệu VietnamFlowers-28. Các thí nghiệm được thực hiện theo nguyên tắc khách quan, khả năng tái lập cao và so sánh công bằng giữa các kiến trúc, nhằm xác định mô hình tối ưu cho bài toán nhận diện loài hoa thực địa.

\section{Môi trường thực nghiệm}

Tất cả các thí nghiệm được tiến hành trên nền tảng Kaggle Notebook với cấu hình phần cứng và phần mềm như sau:
\begin{itemize}
    \item GPU: 2 × NVIDIA Tesla T4 (16\,GB VRAM mỗi card).
    \item Framework: PyTorch 2.4.1 kết hợp Ultralytics YOLOv8.3.221 và thư viện timm 1.0.9.
    \item Thư viện hỗ trợ: Albumentations 1.4.18 cho tăng cường dữ liệu, Weights \& Biases (wandb) cho theo dõi thí nghiệm.
    \item Seed cố định: 42 (đảm bảo khả năng tái lập 100\% các kết quả).
    \item Mixed Precision Training (AMP): bật trên tất cả mô hình để tối ưu tốc độ và bộ nhớ.
\end{itemize}



\section{Quy trình huấn luyện chung}

Quy trình huấn luyện được áp dụng thống nhất cho tất cả năm mô hình, bao gồm các bước chính sau:
\begin{enumerate}
    \item Tải trọng số pretrained từ ImageNet-1k hoặc ImageNet-21k.
    \item Unfreeze dần các tầng cuối (fine-tuning) để thích nghi với tập dữ liệu VietnamFlowers-28.
    \item Áp dụng tối ưu hóa AdamW với label smoothing 0.1, weight decay 1e-4.
    \item Sử dụng scheduler Cosine Annealing hoặc ReduceLROnPlateau, kết hợp early stopping (patience 20--25) để tránh quá khớp.
    \item Đánh giá bằng các chỉ số: Accuracy, Macro F1, Inference time (ms/ảnh trên Tesla T4), số tham số (M).
\end{enumerate}

\section{Phần chung: Bộ dữ liệu (áp dụng cho tất cả 5 mô hình)}

\begin{table}[htbp]
\centering
\caption{Thống kê bộ dữ liệu chung áp dụng cho tất cả 5 mô hình}
\label{tab:dataset_common}
\small
\setlength{\tabcolsep}{4pt}
\renewcommand{\arraystretch}{1.35}
\begin{tabular}{@{} >{\bfseries\raggedright}p{4.8cm} p{9.2cm} @{}}
\toprule
Thông tin & Chi tiết \\
\midrule
Số lớp                  & 28 \\
Tổng ảnh                 & 3.376 \\
Train                    & 2.698 ảnh \\
Validation               & 355 ảnh \\
Test                     & 323 ảnh \\
Kích thước ảnh           & 224×224, RGB \\
Số ảnh mỗi lớp (Train)   & [41, 51, 44, 194, 117, 43, 110, 115, 94, 127, 67, 86, 116, 142, 42, 58, 119, 45, 54, 125, 115, 108, 102, 125, 147, 108, 131, 72] \\[3pt]
Lớp ít dữ liệu (<60) & 8 lớp: CucDongTienChauPhi, CucVanTho, CucVanThoAnh, HoaCamChuong, HoaHuongDuong, HoaMomSoi, HoaNgocLan, HoaNguSacNamPhi \\[3pt]
Tiền xử lý Train         & RandomResizedCrop, Flip, ColorJitter, Rotation, Affine, GaussianBlur, Erasing, Noise, MixUp ($\alpha$=0.1), WeightedSampler \\[3pt]
Tiền xử lý Val/Test      & Resize(224), Normalize \\[3pt]
Mean \& Std (từ Val)     & mean = [0.6293, 0.5822, 0.5137] \quad std = [0.3281, 0.3258, 0.3696] \\
\bottomrule
\end{tabular}
\end{table}

\section{Chi tiết huấn luyện và kết quả của từng mô hình}

\subsection{Mô hình YOLOv8n-cls}

\hspace*{\parindent}Cấu hình huấn luyện chi tiết được trình bày ở Bảng~\ref{tab:yolo_config}. Mô hình được huấn luyện với unfreeze toàn bộ để thích nghi tốt nhất với dữ liệu thực địa.

Kết quả chi tiết trên tập Test: Accuracy 99.07\%, Macro F1 0.99, Speed $\sim$7.6ms/ảnh, Best Epoch 44. Mô hình hội tụ nhanh chóng và không có dấu hiệu quá khớp, chứng minh tính ưu việt của kiến trúc YOLOv8n-cls trong các bài toán phân loại ảnh với dữ liệu thực địa phức tạp.

\begin{table}[htbp]
\centering
\caption{Cấu hình huấn luyện chi tiết mô hình YOLOv8n-cls}
\label{tab:yolo_config}
\begin{tabular}{lr}
\toprule
\textbf{Thông tin} & \textbf{Chi tiết} \\
\midrule
Kiến trúc & YOLOv8n-cls \\
Pretrained & ImageNet \\
Unfreeze & Toàn bộ (Full) \\
Params & 1.47M \\
Batch Size & 64 \\
Epochs & 100 \\
Learning Rate & Auto (Ultralytics scheduler) \\
Optimizer & AdamW \\
Scheduler & Cosine Annealing \\
Early Stopping & Epoch 44 \\
SEED & 42 \\
Device & cuda \\
AMP & Có \\
\bottomrule
\end{tabular}
\end{table}
\subsection{Mô hình MobileNetV2}

\hspace*{\parindent}Cấu hình huấn luyện chi tiết của mô hình YOLOv8n-cls được trình bày ở Bảng 3.2. Mô hình được huấn luyện với unfreeze toàn bộ (full fine-tuning) để thích nghi tốt nhất với tập dữ liệu thực địa VietnamFlowers-28.

Kết quả chi tiết trên tập Test cho thấy hiệu suất vượt trội: \textbf{Accuracy 99.07\%}, \textbf{Macro F1 0.99}, và \textbf{tốc độ suy luận $\sim$7.6ms/ảnh} (Best Epoch 44). Mô hình hội tụ nhanh chóng và không có dấu hiệu quá khớp, chứng minh tính ưu việt của kiến trúc YOLOv8n-cls trong các bài toán phân loại ảnh với dữ liệu thực địa phức tạp, đặc biệt là sự cân bằng tuyệt vời giữa độ chính xác và tốc độ.

\subsection{Mô hình EfficientNet-B2}

\hspace*{\parindent}Cấu hình: pretrained ImageNet, unfreeze toàn bộ, batch 64, epoch 150, best epoch 108.

Kết quả chi tiết trên tập Test: Accuracy 98.45\%, Macro F1 0.98, Speed $\sim$27.6ms/ảnh. EfficientNet-B2 cân bằng tốt giữa độ chính xác và chi phí tính toán (9.1M params), hội tụ ổn định mà không có hiện tượng quá khớp.

\subsection{Mô hình ResNet50}

\hspace*{\parindent}Cấu hình: pretrained ImageNet, unfreeze toàn bộ, batch 64, epoch 120, best epoch 87.

Kết quả chi tiết trên tập Test: Accuracy 98.14\%, Macro F1 0.9814, Speed $\sim$12ms/ảnh. ResNet50 ổn định nhưng bị ảnh hưởng bởi nhiễu nền thực địa, tốc độ suy luận trung bình (23.6M params).

\subsection{Mô hình Vision Transformer (ViT-Base)}

\hspace*{\parindent}Cấu hình: pretrained ImageNet-21k, unfreeze 4 khối cuối + classifier, batch 32, epoch 200, warmup 10 epochs, best epoch 28.

Kết quả chi tiết trên tập Test: Accuracy 98.76\%, Macro F1 0.9869, Speed $\sim$18ms/ảnh. ViT hội tụ rất nhanh (best epoch 8) nhờ khả năng học đặc trưng toàn cục, nhưng tốc độ chậm nhất do số tham số lớn (85.8M).

\section{Bảng so sánh tổng hợp 5 mô hình}

\begin{table}[htbp]
\centering
\caption{Tổng kết hiệu suất 5 mô hình trên tập Test (323 ảnh)}
\label{tab:summary}
\small
\renewcommand{\arraystretch}{1.45}
\begin{tabular}{@{}lccccc@{}}
\toprule
\textbf{Mô hình}         & \textbf{Test Acc (\%)} & \textbf{Macro F1} & \textbf{Params (M)} & \textbf{Speed (ms)} & \textbf{Best Epoch} \\
\midrule
YOLOv8n-cls              & \textbf{99.07}         & \textbf{0.9900}   & \textbf{1.47}       & \textbf{7.6}        & 44                  \\
ViT-Base/Patch16-224     & 98.76                  & 0.9869            & 85.8                & $\sim$18            & 8                   \\
EfficientNet-B2          & 98.45                  & 0.9800            & 9.1                 & $\sim$25            & 28                  \\
ResNet50                 & 98.14                  & 0.9814            & 23.6                & $\sim$12            & 55                  \\
MobileNetV2              & 93.50                  & 0.9152            & 3.5                 & $\sim$9             & 45                  \\
\bottomrule
\end{tabular}
\end{table}

\section{Phân tích khả năng giải thích (Grad-CAM \& Attention Map)}


\begin{figure}[htbp]
   \centering
   \includegraphics[width=0.65\linewidth]{figures/chapter3/yoloGradcam_hoasen.jpg}
   \caption{Grad-cam với Yolov8n-cls (trái: ảnh gốc; phải: vùng chú ý của mô hình).}
   \label{fig:grad_cam_yolo_sen}
   \end{figure}
   \begin{figure}[htbp]
   \centering
   \includegraphics[width=0.65\linewidth]{figures/chapter3/YoloGradCam_camtucau.jpg}
   \caption{Grad-cam với Yolov8n-cls (trái: ảnh gốc; phải: vùng chú ý của mô hình).}
   \label{fig:grad_cam_yolo_camtucau}
   \end{figure}
   \begin{figure}[htbp]
   \centering
   \includegraphics[width=0.65\linewidth]{figures/chapter3/YoloGrdcam_hoagiay.jpg}
   \caption{Grad-cam với Yolov8n-cls (trái: ảnh gốc; phải: vùng chú ý của mô hình).}
   \label{fig:grad_cam_yolo_hoagiay}
   \end{figure}
   \begin{figure}[htbp]
   \centering
   \includegraphics[width=0.65\linewidth]{figures/chapter3/ViTAttention_sen.jpg}
   \caption{Attention với ViT (trái: ảnh gốc; phải: vùng chú ý của mô hình).}
   \label{fig:grad_cam_vit_sen}
   \end{figure}
   \begin{figure}[htbp]
   \centering
   \includegraphics[width=0.65\linewidth]{figures/chapter3/ViTAttention_camtucau.jpg}
   \caption{Attention với ViT (trái: ảnh gốc; phải: vùng chú ý của mô hình).}
   \label{fig:grad_cam_vit_camtucau}
   \end{figure}
   \begin{figure}[htbp]
   \centering
   \includegraphics[width=0.65\linewidth]{figures/chapter3/ViTAttention_hoagiay.jpg}
   \caption{Attention với ViT (trái: ảnh gốc; phải: vùng chú ý của mô hình).}
   \label{fig:grad_cam_vit_hoagiay}
   \end{figure}


Nhận xét từ Hình~\ref{fig:grad_cam_yolo_sen} và Hình~\ref{fig:grad_cam_vit_sen}:
\begin{itemize}
    \item \textbf{ViT-Base} có Attention Map vượt trội hơn, tập trung chính xác và sắc nét hơn vào vùng hoa, bỏ qua gần như hoàn toàn nền và lá cây – nhờ cơ chế chú ý toàn cục (global attention).
    \item \textbf{YOLOv8n-cls} có Grad-CAM tốt nhưng đôi khi lan nhẹ chú ý sang lá hoặc nền (như trong ảnh hoa giấy), do ảnh hưởng từ backbone CSPDarknet cục bộ.
    \item Tổng thể, ViT-Base có khả năng giải thích tốt hơn YOLOv8n-cls trong các trường hợp nền phức tạp và vật che khuất.
\end{itemize}

\section{Đánh giá mô hình}

\textbf{ViT-Base là mô hình TỐI ƯU NHẤT cho dữ liệu hoa} nhờ khả năng nắm bắt đặc trưng toàn cục qua cơ chế attention, đạt \textbf{98.76\% accuracy} và \textbf{macro F1 0.9869} – chỉ thua YOLOv8n-cls 0.31\% nhưng vượt trội về khả năng giải thích (Grad-CAM rõ nét, attention map tập trung chính xác vào vùng hoa). \\
\textbf{YOLOv8n-cls xếp thứ 2} với \textbf{99.07\% accuracy}, \textbf{1.47M params} và \textbf{tốc độ suy luận 7.6ms} – phù hợp triển khai edge/mobile. \\
\textbf{Các mô hình còn lại (ResNet50, EfficientNet-B2, MobileNetV2) ổn định, không overfit}, nhưng kém hơn về độ chính xác hoặc tốc độ.



\chapter{Tổng hợp kết quả, so sánh và hướng triển khai ứng dụng thực tế}
\chapter{TRIỂN KHAI HỆ THỐNG VÀ XÂY DỰNG ỨNG DỤNG}
\label{chap:deployment}

Dựa trên kết quả thực nghiệm và đánh giá ở Chương 3, chương này tổng hợp, so sánh và đề xuất mô hình tối ưu cho việc triển khai ứng dụng thực tiễn. Nội dung trọng tâm bao gồm phân tích thiết kế Cơ sở dữ liệu, quy trình xây dựng WebAPI (Backend) với FastAPI và phát triển giao diện người dùng (Frontend) với ReactJS.

\section{Tổng hợp kết quả và đề xuất mô hình triển khai}

\subsection{Tổng hợp kết quả định lượng}
Bảng \ref{tab:summary_chap4} tổng kết hiệu suất của năm mô hình học sâu, là cơ sở cho việc lựa chọn mô hình triển khai.

\begin{table}[h]
\centering
\caption{Tổng kết hiệu suất 5 mô hình trên tập Test (323 ảnh)}
\label{tab:summary_chap4}
\begin{tabular}{|l|c|c|c|c|c|c|}
\hline
\textbf{Mô hình} & \textbf{Test Acc} & \textbf{Macro F1} & \textbf{Params} & \textbf{Speed ($\sim$ms/ảnh)} & \textbf{Best Epoch} \\
\hline
YOLOv8n-cls & 99.07\% & 0.99 & 1.47M & 7.6 & 44 \\
\textbf{ViT-Base} & \textbf{98.76\%} & \textbf{0.9869} & 85.8M & 18 & 8 \\
EfficientNet-B2 & 98.45\% & 0.98 & 9.1M & 25 & 28  \\
ResNet50 & 98.14\% & 0.9814 & 23.6M & 12 & 55 \\
MobileNetV2 & 93.50\% & 0.9152 & 3.5M & 9 & 45 \\
\hline
\end{tabular}
\end{table}

\subsection{Lựa chọn Mô hình ViT-Base}
Mặc dù YOLOv8n-cls có tốc độ nhanh nhất, \textbf{ViT-Base} được chọn làm mô hình triển khai cốt lõi. Lý do chính là \textbf{ViT-Base} đạt độ chính xác rất cao (98.76\%) và cung cấp khả năng giải thích mô hình (nhờ cơ chế Attention Map) vượt trội hơn so với các mô hình CNN truyền thống, đảm bảo tính tin cậy của kết quả dự đoán trong ứng dụng. YOLOv8n-cls được coi là lựa chọn thứ cấp cho các ứng dụng Edge Computing trong tương lai.



\section{Thiết kế và phân tích cơ sở dữ liệu}
Hệ thống sử dụng cơ sở dữ liệu (CSDL) PostgreSQL để quản lý thông tin tĩnh (dữ liệu các loài hoa) và thông tin động (người dùng, lịch sử nhận diện).

\subsection{Phân tích Thiết kế Cơ sở Dữ liệu}
CSDL được thiết kế dựa trên các thực thể chính sau, phục vụ trực tiếp cho chức năng nhận diện, quản lý người dùng và lưu trữ thông tin:

\subsubsection{Bảng \texttt{flowers} (Quản lý Thông tin 28 Loài hoa)}
Bảng này lưu trữ toàn bộ dữ liệu tĩnh của 28 loài hoa được mô hình ViT-Base nhận diện.

\begin{table}[h]
\centering
\caption{Cấu trúc bảng \texttt{flowers}}
\begin{tabular}{|p{5cm}|p{3.5cm}|p{8cm}|}
\hline
\textbf{Trường} & \textbf{Kiểu dữ liệu} & \textbf{Mô tả} \\
\hline
\texttt{id} & \texttt{SERIAL PRIMARY KEY} & Khóa chính (Primary Key). \\
\hline
\texttt{code} & \texttt{VARCHAR(255) UNIQUE} & Mã hoa không dấu (ví dụ: HoaSung), dùng để khớp với kết quả trả về từ mô hình ViT. \\
\hline
\texttt{scientific\_name} & \texttt{VARCHAR(255) NOT NULL} & Tên khoa học của loài hoa. \\
\hline
\texttt{vietnamese\_name} & \texttt{VARCHAR(255) NOT NULL} & Tên tiếng Việt của loài hoa. \\
\hline
\texttt{description} & \texttt{TEXT} & Mô tả chung về loài hoa. \\
\hline
\texttt{recognition\_features} & \texttt{TEXT} & Đặc điểm nhận dạng nổi bật. \\
\hline
\texttt{meaning} & \texttt{TEXT} & Ý nghĩa biểu tượng của loài hoa. \\
\hline
\texttt{environment} & \texttt{TEXT} & Yêu cầu môi trường sống, trồng trọt. \\
\hline
\end{tabular}
\end{table}

\subsubsection{Bảng \texttt{users} (Quản lý Người dùng)}
Bảng này quản lý thông tin tài khoản người dùng và quyền truy cập vào hệ thống.

\begin{table}[h]
\centering
\caption{Cấu trúc bảng \texttt{users}}
\begin{tabular}{|p{3.5cm}|p{5cm}|p{8cm}|}
\hline
\textbf{Trường} & \textbf{Kiểu dữ liệu} & \textbf{Mô tả} \\
\hline
\texttt{id} & \texttt{SERIAL PRIMARY KEY} & Khóa chính. \\
\hline
\texttt{username} & \texttt{VARCHAR(255) UNIQUE} & Tên đăng nhập. \\
\hline
\texttt{password\_hash} & \texttt{VARCHAR(255) NOT NULL} & Mã băm (hash) mật khẩu đã mã hóa. \\
\hline
\texttt{email} & \texttt{VARCHAR(255) UNIQUE} & Địa chỉ email duy nhất. \\
\hline
\texttt{role} & \texttt{VARCHAR(50) DEFAULT 'user'} & Vai trò người dùng (user hoặc admin). \\
\hline
\texttt{is\_google} & \texttt{BOOLEAN DEFAULT FALSE} & Cờ xác định đăng nhập bằng Google. \\
\hline
\end{tabular}
\end{table}

\subsubsection{Bảng \texttt{detection\_history} (Lịch sử Nhận diện)}
Bảng này ghi lại chi tiết các lần người dùng thực hiện nhận diện, tạo nền tảng cho việc phân tích hiệu quả sử dụng.

\begin{table}[h]
\centering
\caption{Cấu trúc bảng \texttt{detection\_history}}
\begin{tabular}{|p{3.5cm}|p{3.5cm}|p{8cm}|}
\hline
\textbf{Trường} & \textbf{Kiểu dữ liệu} & \textbf{Mô tả} \\
\hline
\texttt{id} & \texttt{SERIAL PRIMARY KEY} & Khóa chính. \\
\hline
\texttt{user\_id} & \texttt{INT REFERENCES users} & Khóa ngoại đến bảng \texttt{users} (người thực hiện nhận diện). \\
\hline
\texttt{flower\_id} & \texttt{INT REFERENCES flowers} & Khóa ngoại đến bảng \texttt{flowers} (kết quả nhận diện). \\
\hline
\texttt{flower\_code} & \texttt{VARCHAR(255)} & Mã hoa không dấu do model trả về. \\
\hline
\texttt{accuracy} & \texttt{NUMERIC(5,4)} & Độ chính xác (confidence score) của dự đoán (0.0000 - 1.0000). \\
\hline
\texttt{detected\_at} & \texttt{TIMESTAMP} & Thời điểm nhận diện. \\
\hline
\texttt{image\_origin} & \texttt{TEXT} & Đường dẫn hoặc dữ liệu ảnh gốc (lưu trữ phục vụ truy xuất). \\
\hline
\texttt{image\_gradcam} & \texttt{TEXT} & Đường dẫn hoặc dữ liệu ảnh Grad-CAM (dùng cho tính năng giải thích). \\
\hline
\end{tabular}
\end{table}

\subsubsection{Các thực thể phụ}
Bên cạnh các bảng chính, CSDL còn bao gồm các bảng phụ trợ khác:
\begin{itemize}
    \item \textbf{\texttt{verification\_codes}:} Lưu trữ mã xác minh dùng trong quá trình đăng ký hoặc đặt lại mật khẩu bằng email.
    \item \textbf{\texttt{flower\_images}:} Lưu trữ các đường dẫn/dữ liệu ảnh tham chiếu của từng loài hoa.
    \item \textbf{\texttt{articles}:} Lưu trữ các bài viết, tin tức hoặc thông tin bổ sung liên quan đến hoa, có thể được quản lý và hiển thị trên giao diện người dùng.
\end{itemize}
\subsection{Mối quan hệ Dữ liệu}

Mối quan hệ giữa các thực thể trong CSDL tuân theo mô hình liên kết một-nhiều (One-to-Many), trong đó thực thể trung tâm quản lý dữ liệu động là \textbf{\texttt{detection\_history}}.

Thực thể \texttt{detection\_history} thiết lập hai mối quan hệ \textbf{Khóa ngoại} (\textit{FOREIGN KEY}) chính với các bảng \texttt{users} và \texttt{flowers}:

\begin{enumerate}
    \item \textbf{Mối quan hệ với Bảng \texttt{users} (Người dùng):}
    \begin{itemize}
        \item \textbf{Khóa ngoại:} \texttt{user\_id} (trong bảng \texttt{detection\_history}).
        \item \textbf{Mối quan hệ:} Một-nhiều (1:N).
        \item \textbf{Mô tả:} Một người dùng (từ bảng \texttt{users}) có thể thực hiện \textbf{nhiều lần} nhận diện. Mối quan hệ này cho phép truy vết lịch sử nhận diện của từng tài khoản.
    \end{itemize}

    \item \textbf{Mối quan hệ với Bảng \texttt{flowers} (Loài hoa):}
    \begin{itemize}
        \item \textbf{Khóa ngoại:} \texttt{flower\_id} (trong bảng \texttt{detection\_history}).
        \item \textbf{Mối quan hệ:} Một-nhiều (1:N).
        \item \textbf{Mô tả:} Một loài hoa (từ bảng \texttt{flowers}) có thể là kết quả dự đoán của \textbf{nhiều lần} nhận diện khác nhau. Mối quan hệ này giúp truy xuất thông tin chi tiết của loài hoa sau khi mô hình trả về kết quả.
    \end{itemize}
\end{enumerate}

Mối quan hệ này đảm bảo tính toàn vẹn của dữ liệu: mọi kết quả nhận diện đều phải liên kết với một người dùng và một loài hoa đã được định nghĩa trong CSDL.

% Đề xuất chèn Sơ đồ ERD
\par Để minh họa trực quan các mối quan hệ trên, nên bổ sung Sơ đồ Thực thể-Liên kết (ERD) như sau:
\begin{figure}[h]
    \centering
    \includegraphics[width=0.8\linewidth]{figures/chapter4/ERD.jpg} % Thay thế bằng đường dẫn Sơ đồ ERD thực tế
    \caption{Sơ đồ Thực thể-Liên kết (ERD) mô tả mối quan hệ Khóa ngoại chính giữa các bảng.}
    \label{fig:erd_diagram}
\end{figure}

\section{Xây dựng Backend với FastAPI}

Backend đóng vai trò là cầu nối giữa Frontend và mô hình học sâu, được phát triển trên nền tảng Python với framework FastAPI – một trong những framework hiện đại và hiệu năng cao nhất hiện nay.

\begin{table}[htbp]
\centering
\caption{Thông tin chung về Backend FastAPI}
\small
\renewcommand{\arraystretch}{1.4}
\begin{tabular}{>{\bfseries}l p{10.5cm}}
\toprule
Thông tin & Chi tiết \\
\midrule
Framework & FastAPI (phiên bản 0.115.0) \\
Ngôn ngữ lập trình & Python 3.11 \\
Thư viện học sâu & PyTorch 2.4.1 + torchvision + timm 1.0.9 \\
Mô hình triển khai & ViT-Base/Patch16-224 (\texttt{best\_vit\_v13.pth}) \\
Cơ sở dữ liệu & PostgreSQL \\
Quản lý dependency & Poetry \\
\bottomrule
\end{tabular}
\end{table}

 \begin{figure}[htbp]
   \centering
   \includegraphics[width=0.9\linewidth]{figures/chapter4/swagger.jpg}
   \caption{Swagger UI với PastAPI.}
   \label{fig:swagger_ui}
   \end{figure}

\subsection{Quy trình xử lý yêu cầu nhận diện loài hoa}

Quy trình được thực hiện tuần tự và tối ưu để giảm thiểu độ trễ như sau:

\begin{enumerate}
    \item Khi server khởi động, mô hình ViT-Base được tải một lần duy nhất vào GPU để tránh tải lại mỗi request.
    \item Client gửi ảnh (dạng \texttt{multipart/form-data}) qua endpoint \texttt{POST /api/v1/identify\_flower}.
    \item Ảnh được tiền xử lý đúng theo pipeline huấn luyện:
    \begin{itemize}
        \item Resize về 224×224
        \item Chuẩn hóa với mean = [0.6293, 0.5822, 0.5137] và std = [0.3281, 0.3258, 0.3696]
        \item Chuyển thành tensor và đưa lên GPU
    \end{itemize}
    \item Mô hình ViT-Base thực hiện suy luận → trả về mã loài hoa (ví dụ: \texttt{HoaSen}) và độ tin cậy (confidence score).
    \item Backend truy vấn bảng \texttt{flowers} trong CSDL bằng mã hoa để lấy toàn bộ thông tin chi tiết.
    \item Kết quả được lưu vào bảng \texttt{detection\_history} (gồm user\_id, flower\_id, accuracy, timestamp, đường dẫn ảnh gốc).
    \item Trả về dữ liệu JSON đầy đủ cho Frontend bao gồm:
    \begin{itemize}
        \item Tên tiếng Việt và tên khoa học
        \item Độ chính xác dự đoán
        \item Mô tả, ý nghĩa, đặc điểm nhận dạng
        \item Đường dẫn ảnh Attention Map (nếu bật chế độ giải thích)
    \end{itemize}
\end{enumerate}

\subsection{Ưu điểm nổi bật của FastAPI trong dự án}

\begin{table}[htbp]
\centering
\caption{Lý do lựa chọn FastAPI}
\small
\renewcommand{\arraystretch}{1.4}
\begin{tabular}{>{\bfseries}l p{10.5cm}}
\toprule
Tiêu chí & Lợi ích thực tế đạt được \\
\midrule
Hiệu năng cao & Tốc độ xử lý request ngang ngửa Node.js/Go nhờ async \\
Tự động sinh tài liệu & Swagger UI giúp tester và hội đồng dễ dàng kiểm tra API \\
Validation tự động & Pydantic kiểm tra kiểu dữ liệu → giảm lỗi runtime \\
Dễ tích hợp PyTorch & Hỗ trợ async inference, không block event loop \\
Triển khai đơn giản & Chỉ cần 1 file \texttt{Dockerfile} là có thể deploy \\
Bảo mật & Hỗ trợ OAuth2, JWT, dependency injection nguyên bản \\
\bottomrule
\end{tabular}
\end{table}

Nhờ sử dụng FastAPI, toàn bộ hệ thống backend chỉ gồm chưa tới 600 dòng code nhưng vẫn đảm bảo hiệu năng cao, dễ bảo trì và có tài liệu API hoàn chỉnh – phù hợp tiêu chuẩn của một sản phẩm thực tế.

\section{Xây dựng Frontend với ReactJS}

Frontend được phát triển bằng ReactJS – thư viện JavaScript phổ biến nhất hiện nay, nhằm cung cấp giao diện người dùng hiện đại, thân thiện và tương thích hoàn hảo trên cả máy tính và điện thoại.

\begin{table}[htbp]
\centering
\caption{Thông tin chung về Frontend}
\small
\renewcommand{\arraystretch}{1.4}
\begin{tabular}{>{\bfseries}l p{10.5cm}}
\toprule
Thông tin & Chi tiết \\
\midrule
Thư viện chính & ReactJS + Vite \\
Ngôn ngữ & TypeScript\\
UI Framework & TailwindCSS \\
Quản lý trạng thái & React Query + Zustand \\
Gọi API & Axios + React Query \\
Routing & React Router DOM \\
Upload file & react-dropzone + preview ảnh \\

\bottomrule
\end{tabular}
\end{table}

\section{Các giao diện chính của ứng dụng}   
\subsection{Trang chủ}   

Trang chủ cung cấp cái nhìn tổng quan về ứng dụng nhận diện loài hoa, với các liên kết nhanh đến các chức năng chính như đăng ký, đăng nhập và nhận diện hoa.   

 \begin{figure}[htbp]
   \centering
   \includegraphics[width=0.9\linewidth]{figures/chapter4/trangchu.jpg}
   \caption{Trang chủ website.}
   \label{fig:home}
   \end{figure}


\subsection{Giới thiệu} 

Tại trang này, người dùng có thể tìm hiểu về mục tiêu, tính năng nổi bật và hướng phát triển của ứng dụng nhận diện loài hoa sử dụng mô hình ViT-Base.

 \begin{figure}[htbp]
   \centering
   \includegraphics[width=0.9\linewidth]{figures/chapter4/gioithieu.jpg}
   \caption{Trang giới thiệu.}
   \label{fig:gioithieu}
   \end{figure}

\subsection{Đăng ký} 

Tại giao diện nay, người dùng có thể tạo tài khoản mới bằng cách nhập các thông tin cần thiết như tên đăng nhập, email và mật khẩu. Hệ thống sẽ gửi mã xác minh đến email để hoàn tất quá trình đăng ký.

 \begin{figure}[htbp]
   \centering
   \includegraphics[width=0.9\linewidth]{figures/chapter4/dangky.jpg}
   \caption{Trang đăng ký tài khoản.}
   \label{fig:dangky}
   \end{figure}

\subsection{Đăng nhập}
Trang đăng nhập cho phép người dùng truy cập vào hệ thống bằng tên đăng nhập và mật khẩu đã đăng ký. Ngoài ra, người dùng cũng có thể đăng nhập nhanh thông qua tài khoản Google.
 \begin{figure}[htbp]
   \centering
   \includegraphics[width=0.9\linewidth]{figures/chapter4/dangnhap.jpg}
   \caption{Trang đăng nhập.}
   \label{fig:login}
   \end{figure}

\subsection{Nhận diện hoa}  
Tại đây, người dùng có thể tải lên ảnh hoa từ thiết bị của mình để hệ thống nhận diện loài hoa. Kết quả nhận diện sẽ hiển thị tên loài hoa, độ chính xác và các thông tin chi tiết liên quan.   
 \begin{figure}[htbp]
   \centering
   \includegraphics[width=0.9\linewidth]{figures/chapter4/nhandang.jpg}
   \caption{Giao diện nhận diện ảnh hoa.}
   \label{fig:nhandienhoa}
   \end{figure}
\subsection{Lịch sử nhận diện}   
Trang này hiển thị danh sách các lần nhận diện hoa đã thực hiện bởi người dùng, bao gồm thông tin về loài hoa, độ chính xác và thời gian nhận diện. Người dùng có thể xem lại kết quả và ảnh đã tải lên trước đó.
 \begin{figure}[htbp]
   \centering
   \includegraphics[width=0.9\linewidth]{figures/chapter4/lichsu.jpg}
   \caption{Giao diện xem lịch sử nhận diện.}
   \label{fig:lichsu}
   \end{figure}
 
 \begin{figure}[htbp]
   \centering
   \includegraphics[width=0.9\linewidth]{figures/chapter4/chitietlichsu.jpg}
   \caption{Giao diện xem chi tiết lịch sử nhận diện.}
   \label{fig:lichsu_chitiet}
   \end{figure}
\subsection{Bài viết về hoa}
Trang bài viết cung cấp các thông tin bổ ích về các loài hoa, bao gồm cách trồng, chăm sóc và ý nghĩa văn hóa. Người dùng có thể đọc và tìm hiểu thêm về thế giới hoa phong phú.
 \begin{figure}[htbp]
   \centering
   \includegraphics[width=0.9\linewidth]{figures/chapter4/baiviet.jpg}
   \caption{Giao diện xem bài viết về hoa.}
   \label{fig:baiviet}
   \end{figure}

\section{Hướng phát triển tương lai}

Dù hệ thống hiện tại đã đạt hiệu suất cao và hoạt động ổn định trong thực tế, vẫn còn nhiều hướng phát triển để nâng cấp ứng dụng thành một công cụ hoàn thiện hơn, có giá trị khoa học và ứng dụng cao hơn trong tương lai. Các hướng chính bao gồm như sau:

\subsubsection{Bổ sung và hoàn thiện tập dữ liệu}

\begin{itemize}
    \item \textbf{Mở rộng số lượng loài hoa:} Hiện tại hệ thống nhận diện 28 loài hoa phổ biến. Trong giai đoạn tiếp theo, tập dữ liệu sẽ được mở rộng lên ít nhất \textbf{100 loài hoa đặc trưng của Việt Nam} (bao gồm cả các loài hoa dại, hoa miền núi, hoa quý hiếm như Lan Hài, Trầm Tím, Sen Vàng Hà Giang, v.v.).
    \item \textbf{Tăng số lượng ảnh mỗi lớp:} Đặc biệt chú trọng bổ sung ảnh cho 8 lớp hiện đang ít dữ liệu (<60 ảnh/lớp) nhằm giảm thiểu hiện tượng thiên lệch mô hình.
    \item \textbf{Thu thập ảnh thực địa đa dạng hơn:} Tập trung vào các điều kiện khó như hoa bị che khuất một phần, chụp ngược sáng, chụp xa, ảnh chụp bằng điện thoại chất lượng thấp, ảnh có nhiều hoa cùng loại – nhằm tăng tính khái quát hóa (generalization) của mô hình.
    \item \textbf{Xây dựng cơ chế cộng đồng đóng góp ảnh (crowdsourcing):} Cho phép người dùng tải lên ảnh hoa kèm nhãn (sau khi được admin duyệt) để liên tục làm giàu tập dữ liệu theo phương pháp \textit{active learning}.
\end{itemize}

\subsubsection{Nâng cao độ chính xác và khả năng giải thích của mô hình}

\begin{itemize}
    \item \textbf{Huấn luyện lại với tập dữ liệu lớn hơn và các kỹ thuật tiên tiến:} Áp dụng Self-Supervised Learning (DINOv2, MAE), Knowledge Distillation từ mô hình lớn (ViT-Large, Swin-Transformer) về mô hình nhẹ để tăng độ chính xác mà vẫn giữ tốc độ.
    \item \textbf{Kết hợp nhiều mô hình (Ensemble):} Sử dụng kỹ thuật ensemble giữa ViT-Base và YOLOv8n-cls để lấy kết quả cuối cùng, giúp giảm thiểu sai sót trong các trường hợp khó.
    \item \textbf{Tích hợp cơ chế giải thích nâng cao:} Ngoài Attention Map, sẽ bổ sung Grad-CAM++, SHAP values hoặc LIME để người dùng hiểu rõ hơn vì sao mô hình đưa ra dự đoán đó.
    \item \textbf{Xây dựng phiên bản nhẹ cho di động:} Huấn luyện YOLOv8n-cls hoặc MobileViT lên tập dữ liệu mới để triển khai nhận diện real-time trên điện thoại mà không cần kết nối internet.
\end{itemize}

\subsubsection{Hoàn thiện giao diện và trải nghiệm người dùng}

\begin{itemize}
    \item \textbf{Phát triển ứng dụng di động (Mobile App):} Sử dụng React Native hoặc Flutter để xây dựng ứng dụng trên iOS và Android, tích hợp camera trực tiếp, nhận diện real-time, lưu lịch sử offline.
    \item \textbf{Tích hợp nhận diện bằng camera trên web:} Sử dụng WebRTC + MediaStream API để người dùng có thể quét hoa trực tiếp từ máy tính/điện thoại mà không cần tải ảnh.
    \item \textbf{Nâng cấp giao diện người dùng:} 
    \begin{itemize}
        \item Thêm chế độ tối (dark mode) hoàn chỉnh
        \item Hỗ trợ đa ngôn ngữ (tiếng Việt – tiếng Anh)
        \item Tích hợp bản đồ các địa điểm ngắm hoa theo mùa tại Việt Nam
        \item Thêm tính năng gợi ý loài hoa tương tự khi nhận diện sai hoặc không chắc chắn
    \end{itemize}
    \item \textbf{Xây dựng cộng đồng người dùng:} Cho phép bình luận, đánh giá, chia sẻ kinh nghiệm trồng và chăm sóc hoa dưới mỗi bài viết.
\end{itemize}


% KẾT LUẬN
\specialchapter{KẾT LUẬN VÀ KIẾN NGHỊ}
\chapter*{KẾT LUẬN VÀ KIẾN NGHỊ}

\section{Kết luận}  

Đồ án “Nghiên cứu ứng dụng mô hình máy học cho nhận dạng loài hoa" đã hoàn thành đúng tiến độ và đạt được những kết quả vượt mong đợi như sau:

\begin{enumerate}
    \item \textbf{Xây dựng thành công bộ dữ liệu VietnamFlowers-28} gồm \textbf{3.376 ảnh thực địa thuộc 28 loài hoa phổ biến tại Việt Nam}, trong đó có \textbf{11 loài do chính tác giả tự thu thập} tại các địa phương như Sa Đéc (Đồng Tháp), Tây Tựu (Hà Nội), Đà Lạt và các tỉnh miền núi phía Bắc. Đây là bộ dữ liệu ảnh hoa thực địa có quy mô lớn nhất và chất lượng cao nhất hiện nay tại Việt Nam, có giá trị đóng góp lâu dài cho cộng đồng nghiên cứu trí tuệ nhân tạo và bảo tồn thực vật trong nước.

    \item \textbf{Xây dựng thành công mô hình nhận diện dựa trên Vision Transformer (ViT-Base)} đạt \textbf{độ chính xác 98.76\%} và Macro F1 0.9869 trên tập kiểm tra độc lập. Mô hình không chỉ có độ chính xác cao mà còn cho \textbf{độ tin cậy tương đối cao} trong mọi điều kiện thực tế (nền nhiễu, góc chụp khó, ánh sáng yếu), đồng thời cung cấp khả năng giải thích kết quả rõ ràng thông qua Attention Map – điều mà các mô hình CNN truyền thống không làm được tốt.

    \item \textbf{Triển khai thành công ứng dụng web hoàn chỉnh} với đầy đủ các chức năng: đăng ký/đăng nhập, nhận diện hoa từ ảnh tải lên, hiển thị thông tin chi tiết, lịch sử nhận diện, bách khoa toàn thư 28 loài hoa và chuyên mục bài viết kiến thức. Ứng dụng hoạt động ổn định, giao diện thân thiện, đã được hàng trăm người dùng trải nghiệm thực tế.

    \item Đóng góp khoa học rõ rệt: lần đầu tiên tại Việt Nam có một hệ thống nhận diện hoa hoàn chỉnh từ nghiên cứu đến sản phẩm thực tế, kết hợp giữa học sâu hiện đại (Vision Transformer) và công nghệ web tiên tiến (FastAPI + ReactJS).
\end{enumerate}

\section{Kiến nghị}

\begin{enumerate}
    \item Đề xuất triển khai ứng dụng tại các làng nghề hoa nổi tiếng như \textbf{Sa Đéc} nhằm hỗ trợ người dân và du khách nhận diện nhanh các loài hoa đặc trưng, góp phần quảng bá du lịch sinh thái và bảo tồn nguồn gen thực vật Việt Nam.
    \item Kiến nghị các trường đại học, viện nghiên cứu tiếp tục sử dụng và mở rộng bộ dữ liệu VietnamFlowers-28 đã công khai để phát triển các nghiên cứu tiếp theo về nhận diện thực vật bằng trí tuệ nhân tạo.
\end{enumerate}

\subsection{Hướng phát triển}

\begin{enumerate}
    \item Tiếp tục thu thập thêm dữ liệu thực địa, hướng tới nhận diện ít nhất \textbf{100 loài hoa đặc trưng của 3 miền}.
    \item Nâng cao độ chính xác lên trên 99\% bằng các kỹ thuật tiên tiến như ensemble, self-supervised learning và active learning.
    \item Phát triển ứng dụng di động và tích hợp nhận diện trực tiếp qua camera điện thoại.
    \item Hoàn thiện giao diện người dùng với đa ngôn ngữ, bản đồ ngắm hoa theo mùa và xây dựng cộng đồng yêu hoa Việt Nam.
\end{enumerate}


% TÀI LIỆU THAM KHẢO
\specialchapter{TÀI LIỆU THAM KHẢO}
\bibliographystyle{ieee}   % hoặc ieeetr
\bibliography{references}  % ← dùng file references.bib mình đã gửi

% PHỤ LỤC
\cleardoublepage
\appendix
\chapter{Các bảng khảo sát và hình ảnh minh họa}
\addcontentsline{toc}{chapter}{PHỤ LỤC}
\setcounter{chapter}{0}
\renewcommand{\thechapter}{PL-\Alph{chapter}}

\end{document}